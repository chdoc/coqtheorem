\documentclass{article}

\usepackage{ntheorem}
\usepackage[hidelinks]{coqtheorem}

\theoremstyle{coqtheorem}
\newtheorem{theorem}{Theorem}
\newtheorem{lemma}[theorem]{Lemma}
\newtheorem{fact}[theorem]{Fact}

% Note that the double braces are mandatory
\setBaseUrl{{http://www.ps.uni-saarland.de/courses/cl-ss16/LectureNotes/html/}}
\setCoqFilename{LectureNotes.Base}

\begin{document}

% Leave the name field empty if you do not want a name
\begin{theorem}[][DM_or]
  DM for or holds.
\end{theorem}

\begin{lemma}[Name]
  Some lemma that is not in Coq.
\end{lemma}

\setCoqFilename{LectureNotes.Chapter1}
\begin{fact}[Boolean and is commutative][andb_com]
  Commutativity for and holds.
\end{fact}

\begin{fact}~
  \begin{enumerate}
  \coqitem[plus_O] $x + 0 = x$
  \coqitem[plus_S] $x + S y = S (x + y)$
  \item Thus $x + y = y + x$ (\coqlink[plus_com]{Link})
\end{enumerate}
\end{fact}

In Theorem \ref{coq:DM_or} we say something about logical or, in
Fact~\ref{coq:andb_com} something about boolean and.

\end{document}



%%% Local Variables:
%%% mode: latex
%%% TeX-master: t
%%% End:
